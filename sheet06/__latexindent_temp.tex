\documentclass[a4paper, 10pt]{article}
    \usepackage[subpreambles=true]{standalone}
    \usepackage[english, american, british]{babel}
    \usepackage[utf8]{inputenc}
    \usepackage[T1]{fontenc}
    \usepackage{hyphenat}
    \hyphenation{Mathe-matik wieder-gewinnen}
    \usepackage{amsmath}
    \usepackage{import}
    \usepackage{tabularx}
    \usepackage{graphicx}
    \usepackage{makecell}
    \usepackage[margin=2cm ]{geometry}

    \title{Einführung in die Softwaretechnik 2018 \\ Sheet 06}
    \author{Maximilian Frühauf}

\begin{document}
\maketitle
\begin{enumerate}
    \item You are designing the access control mechanism for a web-based retail store. 
    Customers browse product information, update their profile, and purchase products. 
    Suppliers can add new products, update product information, and create, process, and examine orders. 
    The store administrator sets the retail prices, makes tailored offers to customers based on their purchasing profiles, 
    and provides marketing services. You have to deal with four actors: StoreAdministrator, Supplier, 
    registered Customer, and unregistered WebUser. Design an access control mechanism for all actors. 
    Customers are already registered. Suppliers need to be registered by the StoreAdministrator. 
    Unregistered web users can only get information about the product and can register by creating the customer info.

    \vspace{0.5cm}

    \begin{table}[h!]
        \centering
        \begin{tabular}{| m{2cm} || c | c | c | c | c | }
            \hline
            Actors / Objects & Product & Order & Offer & customer info& Marketing\\
            \hline
            \hline
            Store Administrator & \makecell{set retail Price} & & make offer &  read customer info& provide marketing services \\
            \hline 
            Supplier & \makecell{<<create>> \\ add product \\ update product info } &  & &  & \\
            \hline 
            registered Customer & \makecell{browse product information  \\ purchase product} &  & & update profile & \\
            \hline 
            unregistered WebUser & browse product information &  && \makecell{<<create>> \\ register user} &\\ 
            \hline 
        \end{tabular}
    \end{table}
    \item
    Given the following sentences, which inheritance mechanisms (specification or implementation inheritance) would be appropriate?
    \begin{itemize}
        \item A Rectangle class inherits from a Polygon class.
        \item A Set class inherits from an existing BinaryTree class.
        \item A Set class inherits from a Bag class (a Bag is defined as an unordered collection).
        \item A Player class inherits from a User class.
    \end{itemize}
    \vspace{0.5cm}
    \item Consider a workflow system supporting software developers. 
    The system enables managers to model the process the developers should follow in 
    terms of processes and work products. The manager assigns specific processes to each developer and sets deadlines 
    for the delivery of each work product. The system supports several types of work products: 
    formatted text, pictures, and URLs. The manager, while editing the workflow, can dynamically set each 
    work product’s type at run-time. 
    Assuming one of your design goals is to design the system so that more work product types can be added in the future, 
    which design pattern would you use to represent work products? 
    Justify your choice and model this design pattern using a UML class diagram.
    \vspace{0.5cm}
    \item
    A server offers a method to perform an encryption. 
    A variety of clients can use this method. At run-time, 
    the server application selects the AES or DES encryption algorithm. 
    AES supports a variety of key lengths with 128, 192, to 256-bit, whereas DES only supports 56-bit keys. 
    The server has to deal with clients that require backward compatibility and clients that require high security. 
    Draw a UML class diagram depicting the classes and methods, and explain your model.
    \vspace{0.5cm}
    \item
    Explain the differences between inheritance and delegation. 
    \vspace{0.5cm}
\end{enumerate}
\end{document}