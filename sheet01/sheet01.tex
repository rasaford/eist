\documentclass[a4paper, 10pt]{article}
    \usepackage[subpreambles=true]{standalone}
    \usepackage[english, american, british]{babel}
    \usepackage[utf8]{inputenc}
    \usepackage[T1]{fontenc}
    \usepackage{hyphenat}
    \hyphenation{Mathe-matik wieder-gewinnen}
    \usepackage{amsmath}
    \usepackage{import}
    \usepackage[margin=2cm ]{geometry}

    \title{Einführung in die Softwaretechnik 2018 \\ Sheet 01}
    \author{Maximilian Frühauf}

\begin{document}
\maketitle

\begin{enumerate}
	\item
	      A functional model describes all the actions the system is
	      able to perform. They can be further refined by in the Use Case diagram.

	      An object model defines the structure of the system.
	      It represents all the required components to implement the functional Model
	      as separate objects, to make the problemset conform to an Object Oriented Programming language.

	      The dynamic model describes the reaction of the system to inputs given by the
		user. As well as the flow of these events through the system.

	\item
	      Software development is difficult, because the problemset to solve is
	      usually not well defined and has to get clarified with the client.
	      This issue made more severe by the immense freedom offered within the discipline of
	      software engineering because every problem can be approached in many different and often incompatible ways.
	      Therefore the amount of possibilities for a given solution is extremely large.

	      Another issue is presented by the fact that software development has to be
	      done in teams and therefore introduces team management issues, like communication
	      difficulties to the process.
	\item
	      Change is inevitable in the software development process because usually not all
	      the requirements of the finished product get defined from the start. Therefore changes have
	      to get made during the development cycle.

	      In my own experience the most common type of changes are those of requirements.
	      This was most apparent when I previously worked on a student project with a few friends.
	      We started the project with only loosely defined our goals and a very rough feature set
	      in the beginning and continued to iterate as we went further along in the development cycle.
	      Therefore we had to change the requirements to our product on a continuous basis.

	      This process also brought with it a lot of technological changes, via the libraries and tools we
	      were using as the chosen project domain was new to all of us. Whenever we hit a wall with any given strategy
	      we had to find another solution and shift the technology stack to accommodate the current requirements.

	      Although we did experience all these changes during the project, we did not have many organizational changes.
	      We kept the hierarchy flat and distributes all tasks among the team member who had the most
	      available free time at the moment.
	\item
	      Modeling is a process done before, or during the development of the
	      actual source code. It describes the project on a higher level and discusses
	      different methods of implementing certain required features.

	      In contrast programming can also concern itself with some level of modelling a problem, but
	      is more focused on the actual implementation as it is usually expressed in a lower level language than
	      the modelling process.
	\item
		A Model is an abstraction of a problem. By ignoring some of it's intricate details, models help in dealing
		with complexity.

		A System is a set of different parts that communicate with each other. It can be described by many individual models, 
		that focus on the parts that comprise the system. All these Models combined are then called the system model. 

		Because a system model or even a single model can get very complicated views are introduced. They allow one to focus 
		only on a specific part of a model to limit the needed mental capacity for understanding a given model.
		Views do not have to be limited to a singe model but can rather, combine parts of any number of different abstractions.
		Therefore it is also possible for Views to overlap in the domain they represent1.
\end{enumerate}
\end{document}