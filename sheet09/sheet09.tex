\documentclass[a4paper, 10pt]{article}
    \usepackage[subpreambles=true]{standalone}
    \usepackage[english, american, british]{babel}
    \usepackage[utf8]{inputenc}
    \usepackage[T1]{fontenc}
    \usepackage{hyphenat}
    \hyphenation{Mathe-matik wieder-gewinnen}
    \usepackage{amsmath}
    \usepackage{import}
    \usepackage{tabularx}
    \usepackage{graphicx}
    \usepackage{makecell}
    \usepackage{verbatim}
    \usepackage{tabularx}
    \usepackage[margin=1cm ]{geometry}

    \title{Einführung in die Softwaretechnik 2018 \\ Sheet 09}
    \author{Maximilian Frühauf}

\begin{document}
\maketitle
\begin{enumerate}
    \item Explain at least 6 (in total) advantages and disadvantages of Scrum in your
    own words. Provide an example for each advantage / disadvantage.
    \vspace{0.5cm}

    \begin{itemize}
        \item Advantages
    \begin{itemize}
        \item Fast iteration times

            As each sprint is further divided in to several smaller daily sprints, 
            it is possible to quickly iterate on an existing backlog item, if changes turn out to 
            be necessary. This can happen because either a better solution is found during a sprint, 
            or because the requirements of the project change. 

            Scrum allows for this because after every sprint the backlog is adjusted by the Scrum Team members
            and the Scrum Master. 
            This can for example be seen in the development of the bumpers game. Here after we implemented the 
            additional display components, we had to exchange these for a newer version with the \verb+AnalogSpeedometer+.
            
        \item No dependency between different stages of development

            As each backlog item is self contained, Scrum does not model any dependencies between the 
            different stages of the development process. This is in contrast to, for example the waterfall model.

            Therefore it is possible to prioritize certain items in the backlog if it becomes necessary. 
            This allows Scrum to be more accepting of changes to the development process itself.
            An example here would be any development cycle in Scrum. If large changes have to be made, it will get
            necessary to update the models of the system from the Analysis phase. This can be easily done in the Scrum process.
        \item Potentially shippable product increments

            After each sprint the result is a potentially shippable product increment, that can be sent to 
            the client for evaluation.
            This allows the client to see the current progress of the development team.
            We saw an example of this in the development of Bumpers, as after every sprint we had produced a
            shippable product increment in the shape of our submission to the Artemis platform.

    \end{itemize}
        \item Disadvantages
        \begin{itemize}
        \item Many Meetings

        Scrum requires more meetings between all the participants of the project than for example the waterfall model. 

        \item Unclear project completion time

        Because Scrum is modeled in cycles a manager cannot as easily as for example in the waterfall model estimate the 
        completion time of the project.

        \item Complexity of the lifecycle model

        Scrum is a more complex model to understand and apply than for example the waterfall of V-Model. Therefore the 
        adoption rate is lower.
        \end{itemize}
    \end{itemize}
    
    \item Name and explain 2 similarities and 2 differences between the Unified
    Process and Scrum.
    \vspace{0.5cm}

    \begin{itemize}
        \item Similarities
        \begin{itemize}
            \item Cyclic model

            Both Unified Process and Scrum are modeled in Cycles. This means that stages of the process repeat and changes can be 
            made to existing work products.
            \item Incremental design 

            Scrum and Unified Process divide the process of software engineering into smaller, more manageable parts. 
            This makes the large concept of developing a system more manageable and allows for better planning of the development process
            as each small part is easier to estimate for. 
        \end{itemize}

        \item Differences
        \begin{itemize}
            \item Concept of stages

            Unified Process splits the Development process into two distinct stages. The Engineering focusing on design and synthesis of concepts 
            and Production stage, which focuses on construction testing and deployment of the product. 
            This distinction is not made within Scrum, where each part of the process is modeled the same way as a sprint.
            \item Handling of frequent change

            Even though Unified Process is a cyclic model it is still build on the iterative idea of stages consisting of distinct phases. 
            Therefore it takes some time to process a change in requirements for the project and propagate these to the actual product itself.

            As Scrum is organized in daily and approximately weekly sprints, change of any type can be easily integrated into the development process 
            by changing the backlog items for the next Sprint. 
        \end{itemize}
    \end{itemize}
\end{enumerate}
\end{document}